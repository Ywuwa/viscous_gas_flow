\documentclass[a4paper]{article}
\usepackage[14pt]{extsizes}
\usepackage[utf8]{inputenc}
\usepackage[T2A]{fontenc}
\usepackage[russian]{babel}
\linespread{1.4}
\usepackage{amsmath, amssymb, amsthm}
\usepackage[pdfencoding=auto,psdextra]{hyperref}
\usepackage{multirow}

\usepackage{graphicx} %Рисунки

\hypersetup{bookmarksnumbered,colorlinks=true,linkcolor=blue,filecolor=blue,urlcolor=blue,
			pdftitle={work},
			pdfproducer={LaTeX with hyperref},pdfcreator={pdflatex}}
 
\urlstyle{same}

\textwidth=500pt
\textheight=700pt
\oddsidemargin=20pt
\hoffset=-1.3cm
\topmargin=-20mm

\usepackage[left=3cm,right=1.5cm,top=2cm,bottom=2cm]{geometry}


\theoremstyle{definition}
\newtheorem{definition}{Определение}
\newtheorem*{example}{Пример}
\newtheorem*{solution}{Решение}
\numberwithin{equation}{section}
\renewcommand{\thesection}{\arabic{section}.}
\renewcommand{\thesubsection}{\arabic{section}.\arabic{subsection}.}
\begin{document}
\begin{titlepage}

\begin{center}


{\Large\bfseries
Решение системы уравнений для вязкого теплопроводного идеального газа\\
Solving the system of equations of a viscous heat-conducting ideal gas\\}
\end{center}


\enlargethispage{4\baselineskip}

\newpage

\end{titlepage}

\title{Решение системы уравнений для вязкого теплопроводного идеального газа}
\author{Тимофеев Кирилл 410}


\righthyphenmin=2
\newpage
\renewcommand{\contentsname}{Содержание}
\tableofcontents
\newpage
\section{Постановка задачи}

Рассмотрим систему уравнений, описывающую движение вязкого теплопроводного газа в двумерном случае:
$$\begin{cases}
\frac{\partial \rho}{\partial t} + div(\rho \textbf{u}) = 0\\
\rho\left[ \frac{\partial \textbf{u}}{\partial t} + (\textbf{u}, \bigtriangledown)\textbf{u} \right] + \bigtriangledown p = L\textbf{u} + \rho f \\
c_v\rho\left[ \frac{\partial\theta}{\partial t} + (\textbf{u},\bigtriangledown \theta) \right] = \kappa\bigtriangleup \theta - p div \textbf{u}- \frac{2}{3}\mu (div \textbf{u})^2 + 2\mu D:D\\
\end{cases}$$

где неизвестными являются: функция скорости $u=u(x,t)$, плотности $\rho=\rho(x,t)$, температуры $\theta=\theta(x,t)$, $(x,t)\in [0, X] \times [0,T]$, где $x=(x_1,x_2)$. Давление записывается выражением $p=R\rho\theta$, R - универсальная газовая постоянная. $\kappa$-теплопроводность, $c_v$-теплоёмкость, $\mu$-коэффицент вязкости являются известными положительными константами. $\kappa=0.023$ кДж/(с*м*К), $c_v=1.3$ кДж/(кг*К), $R=0.00831$ кДж/(моль*К).

$$L\textbf{u}\equiv div(\mu\bigtriangledown \textbf{u}) + \frac{1}{3}\mu\bigtriangledown (div \textbf{u}),$$

D - линейный тензор деформаций Коши-Грина: $D_{ij}=\frac{1}{2}\left( \frac{\partial u_i}{\partial x_j} + \frac{\partial u_j}{\partial x_i}\right)$

Начальные условия: $$(\rho, u, \theta)|_{t=0}=(\rho_0, u_0, \theta_0)$$

Краевые условия: $$u(t,x)=0, \, \theta(t,x)=\tilde{\theta}(t,x), \, (t,x)\in[0,T]\times\partial\Omega.$$

Введём на пространстве следующую сетку:

Разбиение по $[0, X] $: $\omega_{h_k} = \{ mh_k \, | \, m=0,..M_k \}, \, M_kh_k=X_k, \, k=1,2$

Разбиение по $[0, T] $: $\omega_{\tau} = \{n\tau | n=0,..N\}, \quad N\tau=T $

Таким образом сетка на всей области $Q$ задаётся следующим образом: $Q_{\tau,h} = \omega_{\tau} \times \Omega_h = \omega_{\tau}\times\prod\limits_{k = 1}^2 \omega_{h_k}$

Определим ниже скалярные произведения и нормы сеточных функций, которые будут использованы в дальнейшем:
$$(u,v) = \prod\limits_{k=1}^{2}h_k\sum\limits_{x_m\in\Omega_h}u_mv_m, \quad [u,v] = (u,v) + 0.5\prod\limits_{k=1}^{2}h_k\sum\limits_{x_m\in\gamma_h}u_mv_m,$$
$$||v||_{C_h} = \max_{x_m\in\bar{\Omega}_h}|v_m|, \quad ||v||_{L_2}=\sqrt{(v,v)}, \quad |[v]|=\sqrt{[v,v]}, \quad ||v||_2^1 = \sqrt{|[v]|^2 + |v|_1^2}, $$
$$|v|_1 = \sqrt{\prod\limits_{k=1}^2h_k \sum\limits_{k=1}^2\sum\limits_{x_m\in\bar{\Omega}_h}(v_{x_k})^2} $$

Поделим уравнения системы на $\rho$ и сделаем замену $g=\ln\rho$. Получаем:

$\begin{cases}
\frac{\partial g}{\partial t} + (\textbf{u}, \bigtriangledown g) + div\textbf{u}=0\\
\frac{\partial\textbf{u}}{\partial t} + (\textbf{u}, \bigtriangledown)\textbf{u} + \tilde{p}'(g) = e^{-g}L\textbf{u} + f,\\
c_v\left[ \frac{\partial \theta}{\partial t} + (\textbf{u},\bigtriangledown \theta)\right] = e^{-g}\left( \kappa\bigtriangleup \theta - \frac{2}{3}\mu (div \textbf{u})^2 + 2\mu D:D\right) - R\theta div\textbf{u} \\
\end{cases}$\\

где $\tilde{p}'(g) = \frac{\partial p}{\partial \rho}(e^g,\theta)\bigtriangledown g + \frac{\partial p}{\partial \theta}(e^g,\theta)\bigtriangledown \theta$.

Начальные и краевые условия:
$$(g,u,\theta)|_{t=0} = (g_0,u_0,\theta_0),$$
$$u(t,x)=0, \, \theta(t,x)=\tilde{\theta}(t,x), \, (t,x)\in[0,T]\times\partial\Omega.$$

Прежде, чем писать разностные схемы, распишем конфективные слагаемые:
$$(\textbf{u}, \bigtriangledown g) = 0.5((\textbf{u},\bigtriangledown g) + div(g\textbf{u})) - 0.5g (div\textbf{u}),$$
$$(\textbf{u},\bigtriangledown)\textbf{u} = \sum\limits_{k=1}^2 \left[\frac{1}{3}(u_k\frac{\partial u_k}{\partial x_k} + \frac{\partial u_k^2}{\partial x_k}) + \frac{1}{2} \sum\limits_{l=1,l\neq k}^2 (u_l\frac{\partial u_k}{\partial x_l} + \frac{\partial u_l u_k}{\partial x_l} - u_k\frac{\partial u_l}{\partial x_l})\right],$$
$$(\textbf{u},\bigtriangledown \theta) = 0.5((\textbf{u},\bigtriangledown \theta) + div(\theta\textbf{u})) - 0.5\theta(div\textbf{u}) $$
\[D=\left( \begin{matrix}
\frac{\partial u_1}{\partial x_1} & \frac{1}{2}\left( \frac{\partial u_1}{\partial x_2} + \frac{\partial u_2}{\partial x_1}\right) \\
\frac{1}{2}\left( \frac{\partial u_2}{\partial x_1} + \frac{\partial u_1}{\partial x_2}\right) & \frac{\partial u_2}{\partial x_2}\\
\end{matrix}\right) \]
$$D:D = (\frac{\partial u_1}{\partial x_1})^2 + (\frac{\partial u_2}{\partial x_2})^2 + (\frac{\partial u_1}{\partial x_2} + \frac{\partial u_2}{\partial x_1})^2$$
\newpage
\section{Разностная схема}
Теперь можно приступить к написанию разностных схем, сопоставляя неизвестным функциям g, u, $\theta$ сеточные функции G, V, T. В данной работе предлагается частично явная схема, в которой для сеточной функции плотности выполняется условие положительности, а само сеточное решение на каждом временном шаге является решением двух СЛАУ:
$$\begin{cases}
G_t + 0.5\sum\limits_{k=1}^2\left(V_k\hat{G}_{\stackrel{o}{x}_k} + (V_k\hat{G})_{\stackrel{o}{x}_k} - G(V_k)_{\stackrel{o}{x}_k} + 2(\hat{V}_k)_{\stackrel{o}{x}_k}\right) = 0, \quad x\in\Omega_h\\
G_t + 0.5 \left( (V_k\hat{G})_{x_k} - G(V_k)_{x_k} + 2(\hat{V}_k)_{x_k}\right) + \\ \qquad - 0.5 \left( h_k\left( (GV_k)_{x_k\bar{x}_k}^{+1_k} - 0.5(GV_k)_{x_k\bar{x}_k}^{+2_k} + (2-G)((V_k)_{x_k\bar{x}_k}^{+1_k} - 0.5(V_k)_{x_k\bar{x}_k}^{+2_k})\right) \right) = 0, \\ \qquad x\in\gamma_k^-, \, k=1,2\\
G_t + 0.5 \left( (V_k\hat{G})_{x_k} - G(V_k)_{x_k} + 2(\hat{V}_k)_{x_k}\right) + \\ \qquad + 0.5\left( h_k\left( (GV_k)_{x_k\bar{x}_k}^{-1_k} - 0.5(GV_k)_{x_k\bar{x}_k}^{-2_k} + (2-G)((V_k)_{x_k\bar{x}_k}^{-1_k} - 0.5(V_k)_{x_k\bar{x}_k}^{-2_k})\right) \right) = 0, \\ \qquad  x\in\gamma_k^+, \, k=1,2\\
(V_k)_t + \frac{1}{3}\left( V_k(\hat{V}_k)_{\stackrel{o}{x}_k} + (V_k\hat{V}_k)_{\stackrel{o}{x}_k} \right) + \frac{1}{2} \sum\limits_{l=1, l\neq k}^2 \left( V_l(\hat{V}_k)_{\stackrel{o}{x}_l} + (V_l\hat{V}_k)_{\stackrel{o}{x}_l} - V_k(V_l)_{\stackrel{o}{x}_l} \right) + \\ \qquad + RT\hat{G}_{\stackrel{o}{x}_k} + RT_{\stackrel{o}{x}_k} = \tilde{\mu}\left(\frac{4}{3}(\hat{V}_k)_{x_k\bar{x}_k} + \sum\limits_{l=1,l\neq k}^2 (\hat{V}_k)_{x_l\bar{x}_l} \right) - \\ \qquad - (\tilde{\mu} -\mu e^{-G}) \left( \frac{4}{3}(\hat{V}_k)_{x_k\bar{x}_k} + \sum\limits_{l=1,l\neq k}^2 (\hat{V}_k)_{x_l\bar{x}_l} \right) + \frac{\mu e^{-G}}{3} \sum\limits_{l=1,l\neq k}^2 (V_l)_{\stackrel{o}{x}_k\stackrel{o}{x}_l} + f_k, \\ \qquad k=1,2;\quad x\in\Omega_h\\
\hat{V}_k=0, \quad x\in\gamma_h, \quad k=1,2. \\
c_v \left( T_t + 0.5\sum\limits_{k=1}^2 \left( \hat{V}_k\hat{T}_{\stackrel{o}{x}_k} + (\hat{V}_k\hat{T})_{\stackrel{o}{x}_k} - T(\hat{V}_k)_{\stackrel{o}{x}_k} \right) \right) = \\ \qquad = \tilde{\kappa} \sum\limits_{k=1}^2 \left( \hat{T}_{x_k\bar{x}_k}\right) + \sum\limits_{k=1}^2 (\kappa e^{-G} - \tilde{\kappa}) \left( T_{x_k\bar{x}_k}\right) - \\ \qquad - \sum\limits_{k=1}^2 \left( RT (\hat{V}_k)_{\stackrel{o}{x}_k} + \frac{2}{3}\mu e^{-\hat{G}}(\hat{V}_k)_{\stackrel{o}{x}_k}^2 + \frac{2}{3}\mu e^{-\hat{G}} \sum\limits_{l=1,l\neq k}^2 ((\hat{V}_k)_{\stackrel{o}{x}_k} (\hat{V}_l)_{\stackrel{o}{x}_l}) \right) + \\ \qquad + 2\mu e^{-G} \sum\limits_{k=1}^2 \left( (\hat{V}_k)_{\stackrel{o}{x}_k}^2 + 0.5\left( \sum\limits_{l=1,l\neq k}^2 (\hat{V}_k)_{\stackrel{o}{x}_l} + (\hat{V}_l)_{\stackrel{o}{x}_k} \right)^2 \right), \quad x\in\Omega_h,
\end{cases}$$
где $\tilde{\kappa} = ||\kappa e^{-G}||$, $\tilde{\mu} = ||\mu e^{-G}||$.


Уравнения на слой вышеуказанной разностной схемы примут следующий вид:

\begin{eqnarray} \nonumber
& \frac{\hat{G}_{m_1,m_2} - G_{m_1,m_2}}{\tau} + \frac{1}{2} \left( V_{1_{m_1,m_2}} \frac{\hat{G}_{m_1+1,m_2} - \hat{G}_{m_1-1,m_2}}{2h_1} + V_{2_{m_1,m_2}} \frac{\hat{G}_{m_1,m_2+1} - \hat{G}_{m_1,m_2-1}}{2h_2} \right) + \\ \nonumber
& \qquad + \frac{1}{2} \left( \frac{V_{1_{m_1+1,m_2}}\hat{G}_{m_1+1,m_2} - V_{1_{m_1-1,m_2}}\hat{G}_{m_1-1,m_2}}{2h_1} + \frac{V_{2_{m_1,m_2+1}}\hat{G}_{m_1,m_2+1} - V_{1_{m_1,m_2-1}}\hat{G}_{m_1,m_2-1}}{2h_2}\right) - \\ \nonumber
& \qquad - \frac{1}{2}\left( G_{m_1,m_2}\frac{V_{1_{m_1+1,m_2}} - V_{1_{m_1-1,m_2}}}{2h_1} + G_{m_1,m_2}\frac{V_{2_{m_1,m_2+1}} - V_{2_{m_1,m_2-1}}}{2h_2} \right) + \\ 
& \qquad + \left( \frac{\hat{V}_{1_{m_1+1,m_2}} - \hat{V}_{1_{m_1-1,m_2}}}{2h_1} + \frac{\hat{V}_{2_{m_1,m_2+1}} - \hat{V}_{2_{m_1,m_2-1}}}{2h_2} \right) = 0, \\ \nonumber & \quad x=(m_1h_1,m_2h_2)\in \Omega_h
\end{eqnarray}
\begin{eqnarray} \nonumber
& \frac{\hat{G}_{0,m_2} - G_{0,m_2}}{\tau} +  \frac{1}{2} \left( \frac{V_{1_{1,m_2}}\hat{G}_{1,m_2} - V_{1_{0,m_2}}\hat{G}_{0,m_2}}{h_1}\right) - \frac{G_{0,m_2}}{2}\left( \frac{V_{1_{1,m_2}} - V_{1_{0,m_2}}}{h_1} \right) + \left( \frac{\hat{V}_{1_{1,m_2}} - \hat{V}_{1_{0,m_2}}}{h_1} \right) + \\ \nonumber
& \qquad + \frac{h_1}{2}\left( \frac{G_{2,m2}V_{1_{2,m_2}} - 2G_{1,m2}V_{1_{1,m_2}} + G_{0,m2}V_{1_{0,m_2}}}{h_1^2} - 0.5\frac{G_{3,m2}V_{1_{3,m_2}} - 2G_{2,m2}V_{1_{2,m_2}} + G_{1,m2}V_{1_{1,m_2}}}{h_1^2} \right) + \\ 
& \qquad + \frac{h_1}{2}(2-G_{0,m_2})\left( \frac{V_{1_{2,m_2}} - 2V_{1_{1,m_2}} + V_{1_{0,m_2}}}{h_1^2} - 0.5\frac{V_{1_{3,m_2}} - 2V_{1_{2,m_2}} + V_{1_{1,m_2}}}{h_1^2} \right)=0, \\ \nonumber
& \quad \textbf{x}\in\gamma_1^-
\end{eqnarray}
\begin{eqnarray} \nonumber
& \frac{\hat{G}_{M_1,m_2} - G_{M_1,m_2}}{\tau} +  \frac{1}{2} \left( \frac{V_{1_{M_1,m_2}}\hat{G}_{M_1,m_2} - V_{1_{M_1-1,m_2}}\hat{G}_{M_1-1,m_2}}{h_1} \right)  - \frac{G_{M_1,m_2}}{2}\left( \frac{V_{1_{M_1,m_2}} - V_{1_{M_1-1,m_2}}}{h_1}\right) + \\ 
& \qquad + \left( \frac{\hat{V}_{1_{M_1,m_2}} - \hat{V}_{1_{M_1-1,m_2}}}{h_1}\right) + \\ \nonumber
& \qquad + \frac{h_1}{2} \frac{G_{M_1,m2}V_{1_{M_1,m_2}} - 2G_{M_1-1,m2}V_{1_{M_1-1,m_2}} + G_{M_1-2,m2}V_{1_{M_1-2,m_2}}}{h_1^2} - \\ \nonumber
& \qquad - 0.25h_1\frac{G_{M_1-1,m2}V_{1_{M_1-1,m_2}} - 2G_{M_1-2,m2}V_{1_{M_1-2,m_2}} + G_{M_1-3,m2}V_{1_{M_1-3,m_2}}}{h_1^2}  + \\ \nonumber
& + \frac{h_1}{2}(2-G_{M_1,m_2})\left( \frac{V_{1_{M_1,m_2}} - 2V_{1_{M_1-1,m_2}} + V_{1_{M_1-2,m_2}}}{h_1^2} - 0.5\frac{V_{1_{M_1-1,m_2}} - 2V_{1_{M_1-2,m_2}} + V_{1_{M_1-3,m_2}}}{h_1^2} \right)=0, \\ \nonumber
& \quad \textbf{x}\in\gamma_1^+
\end{eqnarray}
\newpage
\begin{eqnarray} \nonumber
& \frac{\hat{G}_{m_1,0} - G_{m_1,0}}{\tau} +  \frac{1}{2} \left(  \frac{V_{2_{m_1,1}}\hat{G}_{m_1,1} - V_{2_{m_1,0}}\hat{G}_{m_1,0}}{h_2}\right) - \frac{G_{m_1,0}}{2}\left(\frac{V_{2_{m_1,1}} - V_{2_{m_1,0}}}{h_2} \right) + \left( \frac{\hat{V}_{2_{m_1,1}} - \hat{V}_{2_{m_1,0}}}{h_2} \right) + \\ \nonumber
& \qquad + \frac{h_2}{2}\left( \frac{G_{m_1,2}V_{2_{m_1,2}} - 2G_{m_1,1}V_{2_{m_1,1}} + G_{m_1,0}V_{2_{m_1,0}}}{h_2^2} - 0.5\frac{G_{m_1,3}V_{2_{m_1,3}} - 2G_{m_1,2}V_{2_{m_1,2}} + G_{m_1,1}V_{2_{m_1,1}}}{h_2^2} \right) + \\ 
& \qquad + \frac{h_2}{2}(2-G_{m_1,0})\left( \frac{V_{2_{m_1,2}} - 2V_{2_{m_1,1}} + V_{1_{m_1,0}}}{h_2^2} - 0.5\frac{V_{2_{m_1,3}} - 2V_{2_{m_1,2}} + V_{2_{m_1,1}}}{h_2^2} \right)=0, \\ \nonumber
& \quad \textbf{x}\in\gamma_2^-
\end{eqnarray}
\begin{eqnarray}\nonumber
& \frac{\hat{G}_{m_1,M_2} - G_{m_1,M_2}}{\tau} +  \frac{1}{2} \left( \frac{V_{2_{m_1,M_2}}\hat{G}_{m_1,M_2} - V_{2_{m_1,M_2-1}}\hat{G}_{m_1,M_2-1}}{h_2}\right) - \frac{G_{m_1,M_2}}{2}\left(\frac{V_{2_{m_1,M_2}} - V_{2_{m_1,M_2-1}}}{h_2} \right) + \\ \nonumber
& \qquad \left( \frac{\hat{V}_{2_{m_1,M_2}} - \hat{V}_{2_{m_1,M_2-1}}}{h_2} \right) + \\
& \qquad + \frac{h_2}{2} \frac{G_{m_1,M_2}V_{2_{m_1,M_2}} - 2G_{m_1,M_2-1}V_{2_{m_1,M_2-1}} + G_{m_1,M_2-2}V_{2_{m_1,M_2-2}}}{h_2^2} - \\ \nonumber
& \qquad - 0.25h_2\frac{G_{m_1,M_2-1}V_{2_{m_1,M_2-1}} - 2G_{m_1M_2-2}V_{2_{m_1,M_2-2}} + G_{m_1,M_2-3}V_{2_{m_1,M_2-3}}}{h_2^2}  + \\ \nonumber
& \qquad + \frac{h_1}{2}(2-G_{m_1,M_2})\left( \frac{V_{2_{m_1,M_2}} - 2V_{2_{m_1,M_2-1}} + V_{2_{m_1,M_2-2}}}{h_2^2} - 0.5\frac{V_{2_{m_1,M_2-1}} - 2V_{2_{m_1,M_2-2,}} + V_{2_{m_1,M_2-3}}}{h_2^2} \right), \\ \nonumber
& \quad \textbf{x}\in\gamma_2^+
\end{eqnarray}
\begin{eqnarray}\nonumber
& \frac{\hat{V}_{1_{m1,m2}} - V_{1_{m1,m2}}}{\tau} + \frac{1}{3}\left( V_{1_{m1,m2}} \frac{\hat{V}_{1_{m1+1,m2}} - \hat{V}_{1_{m1-1,m2}}}{2h_1} + \frac{V_{1_{m1+1,m2}}\hat{V}_{1_{m1+1,m2}} - V_{1_{m1-1,m2}}\hat{V}_{1_{m1-1,m2}}}{2h_1} \right) + \\ \nonumber
& \qquad + \frac{1}{2} \left( V_{2_{m1,m2}}\frac{\hat{V}_{1_{m1,m2+1}} - \hat{V}_{1_{m1,m2-1}}}{2h_2} + \frac{V_{2_{m1,m2+1}}\hat{V}_{1_{m1,m2+1}} - V_{2_{m1,m2-1}}\hat{V}_{1_{m1,m2-1}}}{2h_2}\right) -\\ \nonumber
& \qquad - \frac{1}{2}\left( V_{1_{m1,m2}}\frac{V_{2_{m1,m2+1}} - V_{2_{m1,m2-1}}}{2h_2}\right) + RT_{m1,m2}\frac{\hat{G}_{m1+1,m2} - \hat{G}_{m1-1,m2}}{2h_1} + R\frac{T_{m1+1,m2} - T_{m1-1,m2}}{2h_1} = \\ \nonumber
& \qquad = \tilde{\mu}\left( \frac{4}{3}\frac{\hat{V}_{1_{m1+1,m2}} - 2\hat{V}_{1_{m1,m2}} + \hat{V}_{1_{m1-1,m2}}}{h_1^2} + \frac{\hat{V}_{1_{m1,m2+1}} - 2\hat{V}_{1_{m1,m2}} + \hat{V}_{1_{m1,m2-1}}}{h_2^2} \right) - \\ \nonumber
& \qquad - (\tilde{\mu} - \mu e^{-\hat{G}_{m1,m2}})\left( \frac{4}{3}\frac{V_{1_{m1+1,m2}} - 2V_{1_{m1,m2}} + V_{1_{m1-1,m2}}}{h_1^2} + \frac{V_{1_{m1,m2+1}} - 2V_{1_{m1,m2}} + V_{1_{m1,m2-1}}}{h_2^2} \right) + \\
& \qquad + \frac{\mu e^{-\hat{G}_{m1,m2}}}{3}\frac{V_{2_{m1+1,m2+1}} - V_{2_{m1+1,m2-1}} - V_{2_{m1-1,m2+1}} + V_{2_{m1-1,m2-1}}}{4h_1h_2} + f_{1_{m1,m2}}, \\ \nonumber
& \qquad \textbf{x}\in\Omega_h
\end{eqnarray}
\begin{eqnarray}\nonumber
& \frac{\hat{V}_{2_{m1,m2}} - V_{2_{m1,m2}}}{\tau} + \frac{1}{3}\left( V_{2_{m1,m2}} \frac{\hat{V}_{2_{m1,m2+1}} - \hat{V}_{2_{m1,m2-1}}}{2h_2} + \frac{V_{2_{m1,m2+1}}\hat{V}_{2_{m1,m2+1}} - V_{2_{m1,m2-1}}\hat{V}_{2_{m1,m2-1}}}{2h_2} \right) + \\ \nonumber
& \qquad + \frac{1}{2} \left( V_{1_{m1,m2}}\frac{\hat{V}_{2_{m1+1,m2}} - \hat{V}_{2_{m1-1,m2}}}{2h_1} + \frac{V_{1_{m1+1,m2}}\hat{V}_{2_{m1+1,m2}} - V_{1_{m1-1,m2}}\hat{V}_{2_{m1-1,m2}}}{2h_1}\right) -\\ \nonumber
& \qquad - \frac{1}{2}\left( V_{2_{m1,m2}}\frac{V_{1_{m1+1,m2}} - V_{1_{m1-1,m2}}}{2h_1}\right) + RT_{m1,m2}\frac{\hat{G}_{m1,m2+1} - \hat{G}_{m1,m2-1}}{2h_2} + R\frac{T_{m1,m2+1} - T_{m1,m2-1}}{2h_2} = \\ \nonumber
& \qquad = \tilde{\mu}\left( \frac{4}{3}\frac{\hat{V}_{2_{m1,m2+1}} - 2\hat{V}_{2_{m1,m2}} + \hat{V}_{2_{m1,m2-1}}}{h_2^2} + \frac{\hat{V}_{2_{m1+1,m2}} - 2\hat{V}_{2_{m1,m2}} + \hat{V}_{2_{m1-1,m2}}}{h_1^2}  \right) - \\ \nonumber
& \qquad - (\tilde{\mu} - \mu e^{-\hat{G}_{m1,m2}})\left( \frac{4}{3}\frac{V_{2_{m1,m2+1}} - 2V_{2_{m1,m2}} + V_{2_{m1,m2-1}}}{h_2^2} + \frac{V_{2_{m1+1,m2}} - 2V_{2_{m1,m2}} + V_{2_{m1-1,m2}}}{h_1^2} \right) + \\
& \qquad + \frac{\mu e^{-\hat{G}_{m1,m2}}}{3}\frac{V_{1_{m1+1,m2+1}} - V_{1_{m1+1,m2-1}} - V_{1_{m1-1,m2+1}} + V_{1_{m1-1,m2-1}}}{4h_1h_2} + f_{2_{m1,m2}}, \\ \nonumber
& \qquad \textbf{x}\in\Omega_h
\end{eqnarray}
\begin{eqnarray}\nonumber
& c_v\frac{\hat{T}_{m1,m2} - T_{m1,m2}}{\tau} + 0.5c_v\left( \hat{V}_1\frac{\hat{T}_{m1+1,m2} - \hat{T}_{m1-1,m2}}{2h_1} + \hat{V}_2\frac{\hat{T}_{m1,m2+1} - \hat{T}_{m1,m2-1}}{2h_2}\right) + \\ \nonumber
& \qquad + 0.5c_v\left( \frac{\hat{V}_{1_{m1+1,m2}}\hat{T}_{m1+1,m2} - \hat{V}_{1_{m1-1,m2}}\hat{T}_{m1-1,m2}}{2h_1} +  \frac{\hat{V}_{2_{m1,m2+1}}\hat{T}_{m1,m2+1} - \hat{V}_{2_{m1,m2-1}}\hat{T}_{m1,m2-1}}{2h_2} \right) - \\ \nonumber
& \qquad - 0.5c_vT_{m1,m2}\left( \frac{\hat{V}_{1_{m1+1,m2}} - \hat{V}_{1_{m1-1,m2}}}{2h_1} + \frac{\hat{V}_{2_{m1,m2+1}} - \hat{V}_{2_{m1,m2-1}}}{2h_2}\right) = \\ \nonumber
& \qquad = \tilde{\kappa}\left( \frac{\hat{T}_{m1+1,m2} - 2\hat{T}_{m1,m2} + \hat{T}_{m1-1,m2}}{h_1^2} + \frac{\hat{T}_{m1,m2+1} - 2\hat{T}_{m1,m2} + \hat{T}_{m1,m2-1}}{h_2^2} \right) + \\ \nonumber
& \qquad + (\kappa e^{-\hat{G}_{m1,m2}} - \tilde{\kappa})\left( \frac{T_{m1+1,m2} - 2T_{m1,m2} + T_{m1-1,m2}}{h_1^2} + \frac{T_{m1,m2+1} - 2T_{m1,m2} + T_{m1,m2-1}}{h_2^2} \right) + \\ \nonumber
& \qquad - RT_{m1,m2}\left( \frac{\hat{V}_{1_{m1+1,m2}} - \hat{V}_{1_{m1-1,m2}}}{2h_1} + \frac{\hat{V}_{2_{m1,m2+1}} - \hat{V}_{2_{m1,m2-1}}}{2h_2} \right) - \\ \nonumber
& \qquad - \frac{2}{3}\mu e^{-\hat{G}_{m1,m2}}\left(\left( \frac{\hat{V}_{1_{m1+1,m2}} - \hat{V}_{1_{m1-1,m2}}}{2h_1}\right)^2 + \left( \frac{\hat{V}_{2_{m1,m2+1}} - \hat{V}_{2_{m1,m2-1}}}{2h_2} \right)^2 \right)- \\ \nonumber
& \qquad - \frac{4}{3}\mu e^{-\hat{G}_{m1,m2}}\frac{\hat{V}_{1_{m1+1,m2}} - \hat{V}_{1_{m1-1,m2}}}{2h_1}\frac{\hat{V}_{2_{m1,m2+1}} - \hat{V}_{2_{m1,m2-1}}}{2h_2} + \\ \nonumber
& \qquad + 2\mu e^{-\hat{G}_{m1,m2}}\left(\left( \frac{\hat{V}_{1_{m1+1,m2}} - \hat{V}_{1_{m1-1,m2}}}{2h_1}\right)^2 + \left(\frac{\hat{V}_{2_{m1,m2+1}} - \hat{V}_{2_{m1,m2-1}}}{2h_2} \right)^2\right) +  \\ 
& + 2\mu e^{-\hat{G}_{m1,m2}}\left(\frac{\hat{V}_{2_{m1+1,m2}} - \hat{V}_{2_{m1-1,m2}}}{2h_1} + \frac{\hat{V}_{1_{m1,m2+1}} - \hat{V}_{1_{m1,m2-1}}}{2h_2} \right)^2, \\ \nonumber
& \qquad \textbf{x}\in\Omega_h
\end{eqnarray}
$$\hat{V}_k=0, \, \textbf{x}\in\gamma_k^-, \, \hat{V}_k=0, \, \textbf{x}\in\gamma_k^+, \, k=1,2$$
$$\hat{T}=\theta_{\gamma^-}(\textbf{x},t), \, \textbf{x}\in\gamma^-;\, \hat{T}=\theta_{\gamma^+}(\textbf{x},t), \, \textbf{x}\in\gamma^+$$
Напишем коэффициенты при слагаемых с верхнего слоя и выражение в правых частях из каждого уравнения:
\begin{enumerate}
\item
$\hat{G}_{m1,m2} : \, 1$ \\
$\hat{G}_{m1-1,m2} : \, -\frac{\tau}{4h_1}(V_{1_{m1,m2}} + V_{1_{m1-1,m2}})$\\
$\hat{G}_{m1+1,m2} : \, \frac{\tau}{4h_1}(V_{1_{m1,m2}} + V_{1_{m1+1,m2}})$\\
$\hat{G}_{m1,m2-1} : \, -\frac{\tau}{4h_2}(V_{2_{m1,m2}} + V_{2_{m1,m2-1}})$\\
$\hat{G}_{m1,m2+1} : \, \frac{\tau}{4h_2}(V_{2_{m1,m2}} + V_{2_{m1,m2+1}})$\\
$\hat{V}_{1_{m1-1,m2}} : \, -\frac{\tau}{2h_1}$\\
$\hat{V}_{2_{m1,m2-1}} : \, -\frac{\tau}{2h_2}$\\
$\hat{V}_{1_{m1+1,m2}} : \, \frac{\tau}{2h_1}$\\
$\hat{V}_{2_{m1,m2+1}} : \, \frac{\tau}{2h_2}$\\
$d[m1,m2] : \, G_{m1,m2}+ \frac{\tau}{2}\left( G_{m_1,m_2}\frac{V_{1_{m_1+1,m_2}} - V_{1_{m_1-1,m_2}}}{2h_1} + G_{m_1,m_2}\frac{V_{2_{m_1,m_2+1}} - V_{2_{m_1,m_2-1}}}{2h_2} \right)$
\item 
$\hat{G}_{0,m2} : \, (1-\frac{\tau}{2h_1}V_{1_{0,m2}})$ \\
$\hat{G}_{1,m2} : \, \frac{\tau}{2h_1}V_{1_{1,m2}}$ \\
$\hat{V}_{1_{0,m2}} : \, -\frac{\tau}{h_1}$\\
$\hat{V}_{1_{1,m2}} : \, \frac{\tau}{h_1}$\\
$d[0,m2] : \, G_{0,m2}+\tau\frac{G_{0,m_2}}{2}\left( \frac{V_{1_{1,m_2}} - V_{1_{0,m_2}}}{h_1}\right) -$\\$- \frac{\tau h_1}{2}\left( \frac{G_{2,m2}V_{1_{2,m_2}} - 2G_{1,m2}V_{1_{1,m_2}} + G_{0,m2}V_{1_{0,m_2}}}{h_1^2} - 0.5\frac{G_{3,m2}V_{1_{3,m_2}} - 2G_{2,m2}V_{1_{2,m_2}} + G_{1,m2}V_{1_{1,m_2}}}{h_1^2} \right) - \frac{\tau h_1}{2}(2-G_{0,m_2})\left( \frac{V_{1_{2,m_2}} - 2V_{1_{1,m_2}} + V_{1_{0,m_2}}}{h_1^2} - 0.5\frac{V_{1_{3,m_2}} - 2V_{1_{2,m_2}} + V_{1_{1,m_2}}}{h_1^2} \right) $
\item
$\hat{G}_{M1,m2} : \, (1+\frac{\tau}{2h_1}V_{1_{M1,m2}})$ \\
$\hat{G}_{M1-1,m2} : \, -\frac{\tau}{2h_1}V_{1_{M1-1,m2}}$ \\
$\hat{V}_{1_{M1-1,m2}} : \, -\frac{\tau}{h_1}$\\
$\hat{V}_{1_{M1,m2}} : \, \frac{\tau}{h_1}$\\
$d[M1,m2] : \, G_{M1,m2} + \tau\frac{G_{M_1,m_2}}{2}\left( \frac{V_{1_{M_1,m_2}} - V_{1_{M_1-1,m_2}}}{h_1}\right) -$\\$- \frac{\tau h_1}{2} \frac{G_{M_1,m2}V_{1_{M_1,m_2}} - 2G_{M_1-1,m2}V_{1_{M_1-1,m_2}} + G_{M_1-2,m2}V_{1_{M_1-2,m_2}}}{h_1^2} +$\\$+ 0.25\tau h_1\frac{G_{M_1-1,m2}V_{1_{M_1-1,m_2}} - 2G_{M_1-2,m2}V_{1_{M_1-2,m_2}} + G_{M_1-3,m2}V_{1_{M_1-3,m_2}}}{h_1^2} -$\\$- \frac{\tau h_1}{2}(2-G_{M_1,m_2})\left( \frac{V_{1_{M_1,m_2}} - 2V_{1_{M_1-1,m_2}} + V_{1_{M_1-2,m_2}}}{h_1^2} - 0.5\frac{V_{1_{M_1-1,m_2}} - 2V_{1_{M_1-2,m_2}} + V_{1_{M_1-3,m_2}}}{h_1^2} \right)$
\item 
$\hat{G}_{m1,0} : \, (1 -\frac{\tau}{2h_2} V_{2_{m1,0}})$ \\
$\hat{G}_{m1,1} : \, \frac{\tau}{2h_2}V_{2_{m1,1}}$ \\
$\hat{V}_{2_{m1,0}} : \, -\frac{\tau}{h_2}$\\
$\hat{V}_{2_{m1,1}} : \, \frac{\tau}{h_2}$\\
$d[m1,0] : \, G_{m1,0}+\tau\frac{G_{m_1,0}}{2}\left(\frac{V_{2_{m_1,1}} - V_{2_{m_1,0}}}{h_2} \right) -$\\$- \tau\frac{h_2}{2}\left( \frac{G_{m_1,2}V_{2_{m_1,2}} - 2G_{m_1,1}V_{2_{m_1,1}} + G_{m_1,0}V_{2_{m_1,0}}}{h_2^2} - 0.5\frac{G_{m_1,3}V_{2_{m_1,3}} - 2G_{m_1,2}V_{2_{m_1,2}} + G_{m_1,1}V_{2_{m_1,1}}}{h_2^2} \right) - \tau\frac{h_2}{2}(2-G_{m_1,0})\left( \frac{V_{2_{m_1,2}} - 2V_{2_{m_1,1}} + V_{1_{m_1,0}}}{h_2^2} - 0.5\frac{V_{2_{m_1,3}} - 2V_{2_{m_1,2}} + V_{2_{m_1,1}}}{h_2^2} \right) $
\item 
$\hat{G}_{m1,M2} : \, (1 +\frac{\tau}{2h_2} V_{2_{m1,M2}})$ \\
$\hat{G}_{m1,M2-1} : \, -\frac{\tau}{2h_2}V_{2_{m1,M2-1}}$ \\
$\hat{V}_{2_{m1,M2-1}} : \, -\frac{\tau}{h_2}$\\
$\hat{V}_{2_{m1,M2}} : \, \frac{\tau}{h_2}$\\
$d[m1,M2] : \, G_{m1,M2} + \tau\frac{G_{m_1,M_2}}{2}\left( \frac{V_{2_{m_1,M_2}} - V_{2_{m_1,M_2-1}}}{h_2} \right) -$\\$- \tau\frac{h_2}{2} \frac{G_{m_1,M_2}V_{2_{m_1,M_2}} - 2G_{m_1,M_2-1}V_{2_{m_1,M_2-1}} + G_{m_1,M_2-2}V_{2_{m_1,M_2-2}}}{h_2^2} +$\\$+ 0.25\tau h_2\frac{G_{m_1,M_2-1}V_{2_{m_1,M_2-1}} - 2G_{m_1M_2-2}V_{2_{m_1,M_2-2}} + G_{m_1,M_2-3}V_{2_{m_1,M_2-3}}}{h_2^2} -$\\$- \tau\frac{h_1}{2}(2-G_{m_1,M_2})\left( \frac{V_{2_{m_1,M_2}} - 2V_{2_{m_1,M_2-1}} + V_{2_{m_1,M_2-2}}}{h_2^2} - 0.5\frac{V_{2_{m_1,M_2-1}} - 2V_{2_{m_1,M_2-2,}} + V_{2_{m_1,M_2-3}}}{h_2^2} \right)$
\item
$\hat{V}_{1_{m1,m2}} : \, 1+\frac{8\tau \tilde{\mu}}{3h_1^2} + \frac{2\tau \tilde{\mu}}{h_2^2}$\\
$\hat{V}_{1_{m1+1,m2}} : \, \frac{\tau}{6h_1}V_{1_{m1,m2}} + \frac{\tau}{6h_1}V_{1_{m1+1,m2}} - \frac{4\tau\tilde{\mu}}{3h_1^2}$\\
$\hat{V}_{1_{m1-1,m2}} : \, -\frac{\tau}{6h_1}V_{1_{m1,m2}} - \frac{\tau}{6h_1}V_{1_{m1-1,m2}} - \frac{4\tau\tilde{\mu}}{3h_1^2}$\\
$\hat{V}_{1_{m1,m2+1}} : \, \frac{\tau}{4h_2}V_{2_{m1,m2}} + \frac{\tau}{4h_2}V_{2_{m1,m2+1}} - \frac{\tau\tilde{\mu}}{h_2^2}$\\
$\hat{V}_{1_{m1,m2-1}} : \, -\frac{\tau}{4h_2}V_{2_{m1,m2}} - \frac{\tau}{4h_2}V_{2_{m1,m2-1}} - \frac{\tau\tilde{\mu}}{h_2^2}$\\
$\hat{G}_{m1-1,m2} : \, -RT_{m1,m2}\frac{\tau}{2h_1}$\\
$\hat{G}_{m1+1,m2} : \, RT_{m1,m2}\frac{\tau}{2h_1}$\\
\item
$\hat{V}_{2_{m1,m2}} : \, 1+\frac{8\tau \tilde{\mu}}{3h_2^2} + \frac{2\tau \tilde{\mu}}{h_1^2}$\\
$\hat{V}_{2_{m1,m2+1}} : \, \frac{\tau}{6h_2}V_{2_{m1,m2}} + \frac{\tau}{6h_2}V_{2_{m1,m2+1}} - \frac{4\tau\tilde{\mu}}{3h_2^2}$\\
$\hat{V}_{2_{m1,m2-1}} : \, -\frac{\tau}{6h_2}V_{2_{m1,m2}} - \frac{\tau}{6h_2}V_{2_{m1,m2-1}} - \frac{4\tau\tilde{\mu}}{3h_2^2}$\\
$\hat{V}_{2_{m1+1,m2}} : \, \frac{\tau}{4h_1}V_{1_{m1,m2}} + \frac{\tau}{4h_1}V_{1_{m1+1,m2}} - \frac{\tau\tilde{\mu}}{h_1^2}$\\
$\hat{V}_{2_{m1-1,m2}} : \, -\frac{\tau}{4h_1}V_{1_{m1,m2}} - \frac{\tau}{4h_1}V_{1_{m1-1,m2}} - \frac{\tau\tilde{\mu}}{h_1^2}$\\
$\hat{G}_{m1,m2-1} : \, -RT_{m1,m2}\frac{\tau}{2h_2}$\\
$\hat{G}_{m1,m2+1} : \, RT_{m1,m2}\frac{\tau}{2h_2}$\\
\item
$\hat{T}_{m1,m2} : \, c_v + \frac{2\tau \tilde{\kappa}}{h_1^2} + \frac{2\tau \tilde{\kappa}}{h_2^2}$\\
$\hat{T}_{m1+1,m2} : \, c_v\frac{\tau}{4h_1}(\hat{V}_{1_{m1,m2}} + \hat{V}_{1_{m1+1,m2}}) - \frac{\tau\tilde{\kappa}}{h_1^2}$\\
$\hat{T}_{m1-1,m2} : \, -c_v\frac{\tau}{4h_1}(\hat{V}_{1_{m1,m2}} + \hat{V}_{1_{m1-1,m2}}) - \frac{\tau\tilde{\kappa}}{h_1^2}$\\
$\hat{T}_{m1,m2+1} : \, c_v\frac{\tau}{4h_2}(\hat{V}_{2_{m1,m2}} + \hat{V}_{2_{m1,m2+1}}) - \frac{\tau\tilde{\kappa}}{h_2^2}$\\
$\hat{T}_{m1,m2-1} : \, -c_v\frac{\tau}{4h_2}(\hat{V}_{2_{m1,m2}} + \hat{V}_{2_{m1,m2-1}}) - \frac{\tau\tilde{\kappa}}{h_2^2}$\\
\end{enumerate}
\newpage
Функции $\hat{G}, \hat{V}, \hat{T}$, составляющие решение на верхнем слое, ищутся последовательно как решение двух систем линейных уравнений:

$$\begin{cases}
A^{gv}
\left( \begin{matrix}
\hat{G}\\
\hat{\mathbf{V}}\\
\end{matrix} \right) = 
\left( \begin{matrix}
B_1\\
B_2\\
\end{matrix} \right) \\
A^t \hat{T} = B_t
\end{cases},$$
где матрица $A^{gv}$ является девятидиагональной и умножается на вектор $$(\hat{G},\hat{\mathbf{V}}) = (G_{00},V_{1_{00}},V_{2_{00}},\dots,G_{M_10},V_{1_{M0}},V_{2_{M0}}, \dots, G_{M_1M_2},V_{1_{M_1M_2}},V_{2_{M_1M_2}}),$$
а матрица $A^t$ является пятидиагональной и умножается на вектор $$\hat{T}=(T_{00},T_{10},\dots,T_{M_10},T_{01}, T_{11}, \dots, T_{M_11},\dots, T_{M_1M_2}).$$

Сначала решается уравнение на $G, V$, в котором с нижнего временного слоя будут браться значения T. Далее, используя уже вычисленные значения $\hat{G}, \hat{V}$, ищем решение уравнения на T.

Оба уравнения будут иметь единственное решение т.к. матрица $A^t$ представляется в виде: $A^t=c_vE + c_vA_1^t + \tilde{\kappa} A_2^t$, где $A_1^t$ - кососимметрическая матрица, $A_2^t$ - матрица разностного оператора второй производной в двумерном случае. Таким образом выполнится условие: $(A^tx,x)= (Ex,x)+c_v(A^t_1x,x)+\tilde{\kappa}(A^t_2x,x)\geq(Ex,x)>0, \, \forall x\neq0$. Матрица $A^{gv}$ имеет, с некоторыми усложнениями, аналогичный матрице $A^t$ вид.







\newpage
\section{Отладочный тест}

Для отладки программы, реализующей разностную схему, проверим её на наличие ошибок с помощью задачи, имеющей точное гладкое решение. Здесь и далее расчёты проводятся в области $\Omega = \Omega_{00}\cup \Omega_{10}\cup \Omega_{20}\cup \Omega_{11}\cup \Omega_{21}$.

\begin{center}
\center{\includegraphics[width=0.9\textwidth]{domain.PNG} \\рис.1}\\
\end{center}

На рисунке сплошные отрезки разных цветов соответствуют разным типам границ, а вершины разных цветов - различным типам угловых узлов.
 
Гладкие функции $u_1$, $u_2$, g, $\theta$, использовавшиеся для отладки программы:

$u_1(t,x_1,x_2)=sin(2\pi x_1)sin(2\pi x_2) e^t$

$u_2(t,x_1,x_2)=sin(2\pi x_1)sin(2\pi x_2) e^{-t}$

$g(t,x_1,x_2)=log((cos(2\pi x_1)+\frac{3}{2})(sin(2\pi x_2)+\frac{3}{2})e^t)$

$\theta(t,x_1,x_2)=(cos(3\pi x_1)+\frac{3}{2})(sin(3\pi x_2)+\frac{3}{2})e^t$

Соответственно, к правым частям уравнений добавляются следующие выражения:
\begin{enumerate}
\item
$fg = 1 + e^t sin(2\pi x_1)sin(2\pi x_2)\frac{-2\pi sin(2\pi x_1)}{cos(2\pi x_1) + \frac{3}{2}} + e^{-t} sin(2\pi x_1)sin(2\pi x_2)\frac{2\pi cos(2\pi x_2)}{sin(2\pi x_2) + \frac{3}{2}} +$\\ 
$+ 2\pi e^t cos(2\pi x_1)sin(2\pi x_2) + 2\pi e^{-t}sin(2\pi x_1)cos(2\pi x_2)$
\item
$fu1 = e^t sin(2\pi x_1)sin(2\pi x_2) + $\\
$+ 2\pi e^{2t}sin(2\pi x_1) sin^2(2\pi x_2)cos(2\pi x_1) + 2\pi sin(2\pi x_2) sin^2(2\pi x_1)cos(2\pi x_2) +$\\
$+ Re^t(cos(3\pi x_1)+\frac{3}{2})(sin(3\pi x_2)+\frac{3}{2})\frac{-2\pi sin(2\pi x_1)}{cos(2\pi x_1)+\frac{3}{2}} +$\\
$- 3\pi Re^t sin(3\pi x_1)(sin(3\pi x_2)+\frac{3}{2}) - res,$

где $res1 = \big[ -\frac{4}{3}\mu 4\pi^2 e^t sin(2\pi x_1)sin(2\pi x_2) - \mu 4\pi^2 e^t sin(2\pi x_1)sin(2\pi x_2) +$\\
$+ \frac{1}{3}\mu 4\pi^2 e^{-t} cos(2\pi x_1)cos(2\pi x_2) \big]/res2$

$res2 = e^{g(t,x_1,x_2)},$
\item
$ft = res1-res3/res2-res4/res2 - res5/res2 +res6,$

где $res1 = c_v\big[ (cos(3\pi x_1)+1.5)(sin(3\pi x_2)+1.5)e^t - 3\pi sin(3\pi x_1)(sin(3\pi x_2)+1.5)e^{2t}sin(2\pi x_1)sin(2\pi x_2) + 3\pi(cos(3\pi x_1)+1.5)cos(3\pi x_2)sin(2\pi x_1)sin(2\pi x_2) \big],$

$res2 = e^{g(t,x_1,x_2)},$

$res3 = -9\pi^2\kappa e^t\big[ cos(3\pi x_1)(sin(3\pi x_2)+\frac{3}{2}) - (cos(3\pi x_1)+\frac{3}{2})sin(3\pi x_2) \big]$

$res4 = -\frac{2}{3}\mu 4\pi^2\big[ cos^2(2\pi x_1)sin^2(2\pi x_2)e^{2t} + sin^2(2\pi x_1)cos^2(2\pi x_2)e^{-2t} +$\\$+ 2cos(2\pi x_1)cos(2\pi x_2)sin(2\pi x_1)sin(2\pi x_2) \big]$

$res5 = 2\mu 4\pi^2\big[ (cos(2\pi x_1)sin(2\pi x_2)e^{2t})( cos(2\pi x_1)sin(2\pi x_2)) + $\\
$+ (sin(2\pi x_1)cos(2\pi x_2)e^{-2t})( sin(2\pi x_1)cos(2\pi x_2)) +$\\
$+ 0.5( cos(2\pi x_1)sin(2\pi x_2)e^t+ sin(2\pi x_1)cos(2\pi x_2)e^{-t})( cos(2\pi x_1)sin(2\pi x_2)e^t+ sin(2\pi x_1)cos(2\pi x_2)e^{-t}) \big]$

$res6 = R\theta(t,x_1,x_2)\big[ 2\pi cos(2\pi x_1)sin(2\pi x_2)e^t + 2\pi sin(2\pi x_1)cos(2\pi x_2)e^{-t} \big]$
\end{enumerate}
\newpage

\section{Отладочный тест: результаты в таблицах}
В результате работы программы были получены следующие таблицы норм ошибок для различных $\mu$ (в каждой ячейке записано 3 числа: первое - норма ошибки в С, второе - норма ошибки в $L_2$, третье - время работы программы на данной сетке):

\begin{center}

$\mu=0.1$: 

Таблица для $u_1$
\\[2.0ex]  
\begin{tabular}{|p{0.8in}|p{1.2in}|p{1.2in}|p{1.2in}|} \hline
$\tau\setminus h_1, h_2$ & $h_1=0.05000 ,$ $h_2=0.05000$& $h_1=0.02500 ,$ $h_2=0.02500$& $h_1=0.01250 ,$ $h_2=0.01250$\\ \hline

$0.01250$ & $3.391e-001$ $2.224e-001$ $1.572e+001$ &$2.159e-001$ $1.264e-001$ $1.760e+002$ &$0.000e+000$ $0.000e+000$ $8.310e+002$  \\ \hline
$0.00625$ & $3.164e-001$ $1.816e-001$ $2.309e+001$ &$1.095e-001$ $7.261e-002$ $1.623e+002$ &$1.081e-001$ $5.637e-002$ $2.260e+003$  \\ \hline
$0.00313$ & $3.048e-001$ $1.670e-001$ $3.226e+001$ &$8.687e-002$ $4.920e-002$ $1.806e+002$ &$5.263e-002$ $2.988e-002$ $1.391e+003$  \\ \hline
$0.00156$ & $3.037e-001$ $1.618e-001$ $5.411e+001$ &$7.589e-002$ $3.984e-002$ $2.700e+002$ &$2.798e-002$ $1.747e-002$ $1.643e+003$  \\ \hline
\end{tabular}\\[20pt]
\newpage
Таблица для $u_2$
\\[20pt] 
\begin{tabular}{|p{0.8in}|p{1.2in}|p{1.2in}|p{1.2in}|} \hline
$\tau\setminus h_1, h_2$ & $h_1=0.05000 ,$ $h_2=0.05000$& $h_1=0.02500 ,$ $h_2=0.02500$& $h_1=0.01250 ,$ $h_2=0.01250$\\ \hline

$0.01250$ & $7.500e-002$ $5.556e-002$ $1.572e+001$ &$4.380e-002$ $3.040e-002$ $1.760e+002$ &$0.000e+000$ $0.000e+000$ $8.310e+002$  \\ \hline
$0.00625$ & $6.488e-002$ $4.499e-002$ $2.309e+001$ &$2.298e-002$ $1.753e-002$ $1.623e+002$ &$1.698e-002$ $1.260e-002$ $2.260e+003$  \\ \hline
$0.00313$ & $6.835e-002$ $4.200e-002$ $3.226e+001$ &$1.830e-002$ $1.244e-002$ $1.806e+002$ &$9.000e-003$ $6.809e-003$ $1.391e+003$  \\ \hline
$0.00156$ & $6.987e-002$ $4.112e-002$ $5.411e+001$ &$1.716e-002$ $1.049e-002$ $2.700e+002$ &$5.897e-003$ $4.208e-003$ $1.643e+003$  \\ \hline
\end{tabular}\\[20pt]
\newpage
Таблица для g
 
\begin{tabular}{|p{0.8in}|p{1.2in}|p{1.2in}|p{1.2in}|} \hline
$\tau\setminus h_1, h_2$ & $h_1=0.05000 ,$ $h_2=0.05000$& $h_1=0.02500 ,$ $h_2=0.02500$& $h_1=0.01250 ,$ $h_2=0.01250$ \\ \hline

$0.01250$ & $1.254e+000$ $4.536e-001$ $1.572e+001$ &$8.020e-001$ $2.640e-001$ $1.760e+002$ &$0.000e+000$ $0.000e+000$ $8.310e+002$  \\ \hline
$0.00625$ & $1.225e+000$ $3.715e-001$ $2.309e+001$ &$4.160e-001$ $1.454e-001$ $1.623e+002$ &$4.470e-001$ $1.159e-001$ $2.260e+003$  \\ \hline
$0.00313$ & $1.212e+000$ $3.504e-001$ $3.226e+001$ &$3.446e-001$ $9.449e-002$ $1.806e+002$ &$2.064e-001$ $5.991e-002$ $1.391e+003$  \\ \hline
$0.00156$ & $1.221e+000$ $3.472e-001$ $5.411e+001$ &$3.067e-001$ $7.410e-002$ $2.700e+002$ &$1.198e-001$ $3.373e-002$ $1.643e+003$  \\ \hline
\end{tabular}\\[20pt]

Таблица для $\theta$
  
\begin{tabular}{|p{0.8in}|p{1.2in}|p{1.2in}|p{1.2in}|} \hline
$\tau\setminus h_1, h_2$ & $h_1=0.05000 ,$ $h_2=0.05000$& $h_1=0.02500 ,$ $h_2=0.02500$& $h_1=0.01250 ,$ $h_2=0.01250$ \\ \hline

$0.01250$ & $3.038e+000$ $1.572e+000$ $1.572e+001$ &$2.110e+000$ $7.368e-001$ $1.760e+002$ &$0.000e+000$ $0.000e+000$ $8.310e+002$  \\ \hline
$0.00625$ & $3.023e+000$ $1.536e+000$ $2.309e+001$ &$1.203e+000$ $4.803e-001$ $1.623e+002$ &$1.199e+000$ $3.706e-001$ $2.260e+003$  \\ \hline
$0.00313$ & $2.994e+000$ $1.546e+000$ $3.226e+001$ &$7.032e-001$ $3.890e-001$ $1.806e+002$ &$6.492e-001$ $2.057e-001$ $1.391e+003$  \\ \hline
$0.00156$ & $2.975e+000$ $1.559e+000$ $5.411e+001$ &$6.751e-001$ $3.682e-001$ $2.700e+002$ &$3.396e-001$ $1.332e-001$ $1.643e+003$  \\ \hline
\end{tabular}\\[20pt]
\end{center}
\newpage
\begin{center}

$\mu=0.01$: Таблица для $u_1$

\begin{tabular}{|p{0.8in}|p{1.2in}|p{1.2in}|p{1.2in}|} \hline
$\tau\setminus h_1, h_2$ & $h_1=0.05000 ,$ $h_2=0.05000$& $h_1=0.02500 ,$ $h_2=0.02500$& $h_1=0.01250 ,$ $h_2=0.01250$\\ \hline

$0.01250$ & $4.906e-001$ $3.189e-001$ $1.709e+001$ &$2.310e-001$ $1.111e-001$ $9.365e+001$ &$0.000e+000$ $0.000e+000$ $4.798e+002$  \\ \hline
$0.00625$ & $4.810e-001$ $2.303e-001$ $2.002e+001$ &$1.996e-001$ $6.532e-002$ $8.678e+001$ &$1.550e-001$ $5.198e-002$ $7.202e+002$  \\ \hline
$0.00313$ & $4.906e-001$ $2.091e-001$ $3.641e+001$ &$1.932e-001$ $4.914e-002$ $1.289e+002$ &$7.405e-002$ $2.743e-002$ $6.707e+002$  \\ \hline
$0.00156$ & $5.180e-001$ $1.910e-001$ $5.986e+001$ &$1.922e-001$ $4.474e-002$ $1.921e+002$ &$4.634e-002$ $1.619e-002$ $8.967e+002$  \\ \hline
\end{tabular}\\[20pt]
Таблица для $u_2$

\begin{tabular}{|p{0.8in}|p{1.2in}|p{1.2in}|p{1.2in}|} \hline
$\tau\setminus h_1, h_2$ & $h_1=0.05000 ,$ $h_2=0.05000$& $h_1=0.02500 ,$ $h_2=0.02500$& $h_1=0.01250 ,$ $h_2=0.01250$\\ \hline

$0.01250$ & $9.064e-002$ $8.098e-002$ $1.709e+001$ &$5.769e-002$ $2.876e-002$ $9.365e+001$ &$0.000e+000$ $0.000e+000$ $4.798e+002$  \\ \hline
$0.00625$ & $7.414e-002$ $6.253e-002$ $2.002e+001$ &$3.915e-002$ $1.903e-002$ $8.678e+001$ &$3.235e-002$ $1.332e-002$ $7.202e+002$  \\ \hline
$0.00313$ & $7.533e-002$ $5.909e-002$ $3.641e+001$ &$3.366e-002$ $1.524e-002$ $1.289e+002$ &$1.817e-002$ $7.562e-003$ $6.707e+002$  \\ \hline
$0.00156$ & $8.258e-002$ $5.682e-002$ $5.986e+001$ &$3.368e-002$ $1.394e-002$ $1.921e+002$ &$1.097e-002$ $4.905e-003$ $8.967e+002$  \\ \hline
\end{tabular}\\[20pt]
\newpage
Таблица для g
 
\begin{tabular}{|p{0.8in}|p{1.2in}|p{1.2in}|p{1.2in}|} \hline
$\tau\setminus h_1, h_2$ & $h_1=0.05000 ,$ $h_2=0.05000$& $h_1=0.02500 ,$ $h_2=0.02500$& $h_1=0.01250 ,$ $h_2=0.01250$ \\ \hline

$0.01250$ & $8.337e+000$ $1.030e+000$ $1.709e+001$ &$4.300e+000$ $4.793e-001$ $9.365e+001$ &$0.000e+000$ $0.000e+000$ $4.798e+002$  \\ \hline
$0.00625$ & $7.367e+000$ $9.047e-001$ $2.002e+001$ &$2.547e+000$ $2.873e-001$ $8.678e+001$ &$1.199e+000$ $1.558e-001$ $7.202e+002$  \\ \hline
$0.00313$ & $7.884e+000$ $9.019e-001$ $3.641e+001$ &$1.772e+000$ $2.087e-001$ $1.289e+002$ &$6.096e-001$ $8.368e-002$ $6.707e+002$  \\ \hline
$0.00156$ & $8.341e+000$ $9.467e-001$ $5.986e+001$ &$1.508e+000$ $1.842e-001$ $1.921e+002$ &$4.192e-001$ $5.282e-002$ $8.967e+002$  \\ \hline
\end{tabular}\\[20pt]
Таблица для $\theta$

\begin{tabular}{|p{0.8in}|p{1.2in}|p{1.2in}|p{1.2in}|} \hline
$\tau\setminus h_1, h_2$ & $h_1=0.05000 ,$ $h_2=0.05000$& $h_1=0.02500 ,$ $h_2=0.02500$& $h_1=0.01250 ,$ $h_2=0.01250$ \\ \hline

$0.01250$ & $2.744e+000$ $1.333e+000$ $1.709e+001$ &$1.757e+000$ $5.688e-001$ $9.365e+001$ &$0.000e+000$ $0.000e+000$ $4.798e+002$  \\ \hline
$0.00625$ & $2.837e+000$ $1.409e+000$ $2.002e+001$ &$8.977e-001$ $3.837e-001$ $8.678e+001$ &$1.238e+000$ $2.883e-001$ $7.202e+002$  \\ \hline
$0.00313$ & $2.861e+000$ $1.457e+000$ $3.641e+001$ &$6.701e-001$ $3.445e-001$ $1.289e+002$ &$5.822e-001$ $1.523e-001$ $6.707e+002$  \\ \hline
$0.00156$ & $2.874e+000$ $1.490e+000$ $5.986e+001$ &$6.542e-001$ $3.481e-001$ $1.921e+002$ &$2.805e-001$ $1.000e-001$ $8.967e+002$  \\ \hline
\end{tabular}\\[20pt]
\end{center}
\newpage
\begin{center}

$\mu=0.001$: Таблица для $u_1$

\begin{tabular}{|p{0.8in}|p{1.2in}|p{1.2in}|p{1.2in}|} \hline
$\tau\setminus h_1, h_2$ & $h_1=0.05000 ,$ $h_2=0.05000$& $h_1=0.02500 ,$ $h_2=0.02500$& $h_1=0.01250 ,$ $h_2=0.01250$\\ \hline

$0.01250$ & $8.488e-001$ $9.445e-001$ $2.531e+001$ &$0.000e+000$ $0.000e+000$ $5.190e+002$ &$0.000e+000$ $0.000e+000$ $1.335e+003$  \\ \hline
$0.00625$ & $8.684e-001$ $9.625e-001$ $3.565e+001$ &$2.794e-001$ $7.005e-002$ $7.374e+001$ &$2.119e-001$ $6.755e-002$ $5.373e+002$  \\ \hline
$0.00313$ & $8.769e-001$ $9.704e-001$ $6.287e+001$ &$2.180e-001$ $4.772e-002$ $1.037e+002$ &$9.314e-002$ $3.229e-002$ $4.528e+002$  \\ \hline
$0.00156$ & $8.755e-001$ $9.677e-001$ $1.183e+002$ &$1.882e-001$ $4.319e-002$ $1.859e+002$ &$6.364e-002$ $1.686e-002$ $7.523e+002$  \\ \hline
\end{tabular}\\[20pt]
Таблица для $u_2$

\begin{tabular}{|p{0.8in}|p{1.2in}|p{1.2in}|p{1.2in}|} \hline
$\tau\setminus h_1, h_2$ & $h_1=0.05000 ,$ $h_2=0.05000$& $h_1=0.02500 ,$ $h_2=0.02500$& $h_1=0.01250 ,$ $h_2=0.01250$\\ \hline

$0.01250$ & $2.215e-001$ $2.428e-001$ $2.531e+001$ &$0.000e+000$ $0.000e+000$ $5.190e+002$ &$0.000e+000$ $0.000e+000$ $1.335e+003$  \\ \hline
$0.00625$ & $2.189e-001$ $2.414e-001$ $3.565e+001$ &$5.841e-002$ $2.495e-002$ $7.374e+001$ &$5.940e-002$ $1.919e-002$ $5.373e+002$  \\ \hline
$0.00313$ & $2.177e-001$ $2.406e-001$ $6.287e+001$ &$5.782e-002$ $1.996e-002$ $1.037e+002$ &$3.529e-002$ $1.065e-002$ $4.528e+002$  \\ \hline
$0.00156$ & $2.159e-001$ $2.382e-001$ $1.183e+002$ &$5.177e-002$ $1.820e-002$ $1.859e+002$ &$1.938e-002$ $6.411e-003$ $7.523e+002$  \\ \hline
\end{tabular}\\[20pt]
\newpage
Таблица для g
 
\begin{tabular}{|p{0.8in}|p{1.2in}|p{1.2in}|p{1.2in}|} \hline
$\tau\setminus h_1, h_2$ & $h_1=0.05000 ,$ $h_2=0.05000$& $h_1=0.02500 ,$ $h_2=0.02500$& $h_1=0.01250 ,$ $h_2=0.01250$ \\ \hline

$0.01250$ & $5.199e+001$ $3.420e+000$ $2.531e+001$ &$0.000e+000$ $0.000e+000$ $5.190e+002$ &$0.000e+000$ $0.000e+000$ $1.335e+003$  \\ \hline
$0.00625$ & $9.421e+001$ $5.210e+000$ $3.565e+001$ &$6.330e+000$ $4.978e-001$ $7.374e+001$ &$4.808e+000$ $3.083e-001$ $5.373e+002$  \\ \hline
$0.00313$ & $1.335e+002$ $7.121e+000$ $6.287e+001$ &$3.287e+000$ $2.901e-001$ $1.037e+002$ &$2.279e+000$ $1.626e-001$ $4.528e+002$  \\ \hline
$0.00156$ & $1.623e+002$ $8.528e+000$ $1.183e+002$ &$2.378e+000$ $2.365e-001$ $1.859e+002$ &$1.161e+000$ $8.999e-002$ $7.523e+002$  \\ \hline
\end{tabular}\\[20pt]
Таблица для $\theta$

\begin{tabular}{|p{0.8in}|p{1.2in}|p{1.2in}|p{1.2in}|} \hline
$\tau\setminus h_1, h_2$ & $h_1=0.05000 ,$ $h_2=0.05000$& $h_1=0.02500 ,$ $h_2=0.02500$& $h_1=0.01250 ,$ $h_2=0.01250$ \\ \hline

$0.01250$ & $3.850e+000$ $2.683e+000$ $2.531e+001$ &$0.000e+000$ $0.000e+000$ $5.190e+002$ &$0.000e+000$ $0.000e+000$ $1.335e+003$  \\ \hline
$0.00625$ & $4.111e+000$ $2.822e+000$ $3.565e+001$ &$1.388e+000$ $4.098e-001$ $7.374e+001$ &$1.662e+000$ $3.356e-001$ $5.373e+002$  \\ \hline
$0.00313$ & $4.203e+000$ $2.859e+000$ $6.287e+001$ &$8.085e-001$ $3.561e-001$ $1.037e+002$ &$8.553e-001$ $1.715e-001$ $4.528e+002$  \\ \hline
$0.00156$ & $4.224e+000$ $2.861e+000$ $1.183e+002$ &$6.828e-001$ $3.573e-001$ $1.859e+002$ &$4.267e-001$ $1.058e-001$ $7.523e+002$  \\ \hline
\end{tabular}\\[20pt]
\end{center}
\newpage
Из таблиц можно сделать вывод, что решение РС сходится с регламентированной скоростью  $O(\tau + h^2)$. Также из таблиц видно, что из-за отчасти явного характера схемы существуют некоторые ограничения на шаги сетки по времени и по пространству, однако для получения желаемой точности решения, можно при фиксированном шаге h по пространству достаточно существенно уменьшать шаг по времени $\tau$, что лишь линейно повлияет на время работы программы.
\newpage
\section{Вложенные сетки}
Для получения ответа на вопрос о точности решения, полученного на сетке $Q_{\tau,\textbf{h}}$, в случае, когда точное решение неизвестно, используют расчёты на вложенных сетках $Q_{\tau/2^k,\textbf{h}/2^k}$. В данном случае точная функция известна, и вышеуказанные таблицы позволяют нам судить о том, что сходимость РС есть, однако в качестве дополнительной проверки, произведём расчёты на вложенных сетках.

Ниже в таблицах, в столбцах 1, 2, 3 представлены значения норм ошибок решения, полученного на сетках $Q_{\tau/2,\textbf{h}/2}$, $Q_{\tau/4,\textbf{h}/4}$ и $Q_{\tau/8,\textbf{h}/8}$ соответственно. В первом столбце представлены нормы ошибок для оригинальной сетки $Q_{\tau,\textbf{h}}$.
\newpage
\begin{center}
$\mu=0.1$: Таблица для $u_1$
%\\[2.0ex]  
\begin{tabular}{|p{1.1in}|p{1.2in}|p{1.2in}|p{1.2in}|} \hline
$k\setminus \tau, h $ & 1 & 2 & original \\ \hline

$\tau=0.01250 ,$ $h=0.05000$ &$2.450e-001$ $6.677e-002$ &$3.058e-001$ $8.518e-002$ &$3.391e-001$ $2.227e-001$ \\ \hline
$\tau=0.00625 ,$ $h=0.02500$ &$7.020e-002$ $1.923e-002$ &$9.343e-002$ $2.636e-002$ &$1.095e-001$ $7.263e-002$ \\ \hline
\end{tabular}\\[20pt]
\end{center}
\begin{center}
Таблица для $u_2$
%\\[2.0ex]  
\begin{tabular}{|p{1.1in}|p{1.2in}|p{1.2in}|p{1.2in}|} \hline
$k\setminus \tau, h $ & 1 & 2 & original \\ \hline

$\tau=0.01250 ,$ $h=0.05000$ &$5.387e-002$ $1.699e-002$ &$6.729e-002$ $2.161e-002$ &$7.500e-002$ $5.563e-002$  \\ \hline
$\tau=0.00625 ,$ $h=0.02500$ &$1.437e-002$ $4.833e-003$ &$1.915e-002$ $6.499e-003$ &$2.298e-002$ $1.754e-002$  \\ \hline
\end{tabular}\\[20pt]
\end{center}
\begin{center}
Таблица для g
%\\[2.0ex]  
\begin{tabular}{|p{1.1in}|p{1.2in}|p{1.2in}|p{1.2in}|} \hline
$k\setminus \tau, h $ & 1 & 2 & original \\ \hline

$\tau=0.01250 ,$ $h=0.05000$ &$8.546e-001$ $1.313e-001$ &$1.097e+000$ $1.672e-001$ &$1.254e+000$ $4.541e-001$  \\ \hline
$\tau=0.00625 ,$ $h=0.02500$ &$2.513e-001$ $3.639e-002$ &$3.440e-001$ $5.052e-002$ &$4.160e-001$ $1.455e-001$  \\ \hline
\end{tabular}\\[20pt]
\end{center}
\begin{center}
Таблица для $\theta$
%\\[2.0ex]  
\begin{tabular}{|p{1.1in}|p{1.2in}|p{1.2in}|p{1.2in}|} \hline
$k\setminus \tau, h $ & 1 & 2 & original \\ \hline

$\tau=0.01250 ,$ $h=0.05000$ &$2.424e+000$ $5.240e-001$ &$2.860e+000$ $6.339e-001$ &$3.038e+000$ $1.574e+000$  \\ \hline
$\tau=0.00625 ,$ $h=0.02500$ &$6.955e-001$ $1.370e-001$ &$9.408e-001$ $1.777e-001$ &$1.203e+000$ $4.805e-001$  \\ \hline
\end{tabular}\\[20pt]
\end{center}

Как видно из таблиц, нормы ошибок на вложенных сетках оценивают нормы ошибок, полученных на точном решении, снизу, причём эта оценка улучшается с возрастанием параметра k, в чём и требовалось убедиться.
\newpage
\section{Задача протекания}
\textbf{Постановка задачи}

Напомним, что расчёты производятся в области $\Omega = \Omega_{00}\cup \Omega_{10}\cup \Omega_{20}\cup \Omega_{11}\cup \Omega_{21}$ (рис.2). В область $\Omega$ через границу AF набегает поток, имеющий отличные от газа внутри области характеристики: плотность, скорость и температуру. Отметим, что вследствие этого несколько меняются значения статусов границ, подробнее об этом будет изложено ниже.

\begin{center}
\center{\includegraphics[width=0.9\textwidth]{domain2.PNG} \\рис.2}\\
\end{center}
\newpage
Важным в данном случае является изменение начальных и краевых условий:
\begin{enumerate}
\item $\textbf{u}(0,\textbf{x})=g(0,\textbf{x})=0$,
\item{Характеристики набегающего потока}
\begin{enumerate}
\item $u_1(t,0,x_2)=\tilde{v}=const>0$, $u_2(t,0,x_2)=0$, 
\item $g(t,0,x_2)=\tilde{g}=const>1$,
\item $\theta(t,0,x_2)=\tilde{\theta}=const>293(K), 0\le x_2 \le 1$

\end{enumerate}
\item
\begin{enumerate}
\item $u_1(t,0,x_2)=0$, $u_2(t,0,x_2)=0$, 
\item $g(t,0,x_2)=0$,
\item $\theta(t,0,x_2)=293(K), 1< x_2 \le 2$

\end{enumerate}
\item $\frac{\partial u_1}{\partial x_1}|_{x_1=X_1}=0$
\item $u_1(t,x_1,X_2)=0$, $u_1(t,x1,0)=0$, $u_2(t,\textbf{X})=0$, $u_2(t,x1,0)=0$,

\item $\frac{\partial \theta}{\partial \vec{n}}|_{x_2=0,\textbf{x}=\textbf{X}}=0$ - характеризует теплоизолированность "верхних" и "нижних" стенок, стабилизацию температуры на правой границе.
\end{enumerate}

Таким образом, левые границы области становятся различными по своей сути - через границу AF газ втекает, а на границе DE выполняются условия прилипания. 
\newpage
Вычисления будут производиться до момента времени $N\tau$, для которого $||(G^{N\tau},\textbf{V}^{N\tau},T^{N\tau})-(G^{n_{st}},\textbf{V}^{n_{st}},T^{n_{st}})||_C<\epsilon$, где $\epsilon$ определяется опытным путём. Ввиду изменившихся начально-краевых условий, схема также претерпит модификации:
\begin{enumerate}
\item Из (4),(6):
$$\hat{V}_{1_{M1,m2}} - \hat{V}_{1_{M1-1,m2}}=0, \, m2=0,..M_2$$
$$\hat{T}_{M1,m2} - \hat{T}_{M1-1,m2}=0, \, m2=0,..M_2$$
$$\hat{T}_{m1,1} - \hat{T}_{m1,0}=0, \, m1=0,..M_1$$
$$\hat{T}_{m1,M2} - \hat{T}_{m1,M2-1}=0, \, m1=0,..M_1$$

\item Из условиq (2),(4) следует исчезновение уравнения (2) и замена разностного уравнения (3):
$$G_t+\left( V_{1_{M1,m2}} \frac{\hat{G}_{M1,m2}-\hat{G}_{M1-1,m2}}{h_1} \right) = 0, \, m2=0,..M_2$$

\end{enumerate}
\newpage
\section{Задача протекания: результаты в таблицах}
\begin{center}
Таблицы с временем стабилизации функций, $\tau=0.0125, \, \textbf{h}=0.025$

$\mu=0.1, \, \epsilon=0.1$
\\[2.0ex]  

\begin{tabular}{|p{0.8in}|p{0.6in}|p{1.0in}|p{1.0in}|p{1.0in}|p{1.0in}|} \hline
\multirow{1}{*}{} & $\tilde{v}\setminus \tilde{g} $ & 0.5 & 1 & 1.5 & 2\\ \hline
\multirow{2}{*}{$\tilde{\theta}=293$} & $1$& $5.026e+001$ &$4.401e+001$ &$6.276e+001$ &$1.878e+002$ \\ \cline{2-6}
& $2$& $1.901e+001$ &$5.026e+001$ &$0.000e+000$ &$3.754e+002$  \\ \hline
\multirow{2}{*}{$\tilde{\theta}=303$} & $1$& $7.526e+001$ &$8.776e+001$ &$6.276e+001$ &$5.651e+001$ \\ \cline{2-6}
& $2$& $8.776e+001$ &$7.526e+001$ &$7.526e+001$ &$5.026e+001$  \\ \hline
\multirow{2}{*}{$\tilde{\theta}=313$} & $1$& $1.128e+002$ &$1.253e+002$ &$1.253e+002$ &$2.503e+002$ \\ \cline{2-6}
& $2$& $1.253e+002$ &$1.878e+002$ &$0.000e+000$ &$3.754e+002$  \\ \hline

\end{tabular}\\[20pt]

\end{center}

Из таблицы прослеживается, что время стабилизации возрастает с увеличением $\tilde{\theta}$ (температуры набегающего потока). Также время стабилизации преимущественно возрастает с ростом значения $\tilde{g}$ (логарифма плотности набегающего потока). Чёткой зависимости от $\tilde{v}$ (скорости набегающего потока) не прослеживается, хотя в большинстве случаев видно, что с увеличением этого параметра время стабилизации увеличивается.

\newpage
Ниже приведены кадры из созданной в рамках данной курсовой видео-модели течения газа. На этих кадрах изображены тепловые карты газа в области, однако эти карты также наглядно показывают по какой траектории движется газ в области.
\begin{center}
\center{\includegraphics[width=0.9\textwidth]{mu01_v2g2t303_T1.PNG}}\\
\center{\includegraphics[width=0.9\textwidth]{mu01_v2g2t303_T31.PNG}}\\
\end{center}
\begin{center}
\center{\includegraphics[width=0.75\textwidth]{mu01_v2g2t303_T96.PNG}}\\
\center{\includegraphics[width=0.75\textwidth]{mu01_v2g2t303_T521.PNG}}\\
\center{\includegraphics[width=0.75\textwidth]{mu01_v2g2t303_T3036.PNG}}\\
\end{center}

\newpage
\section{Заключение}
По результатам, полученным в ходе работы, можно сделать вывод, что описанная РС может быть использована для моделирования нестационарного течения газа в областях непрямоугольной формы, причём получаемые результаты обладают высоким порядком точности $O(\tau + h^2)$. Конечно, в силу отчасти явного характера РС (при подсчёте скорости-плотности используются значения температуры с нижнего слоя) существуют ограничения на шаги $\tau, \, h$. Помимо прочего, для задач протекания, по мнению автора, требуется более педантичный подход к физической подоплёке процесса. Например, численные эксперименты показали, что при сильно малых значениях параметра вязкости ($\mu=10^{-4}$) решение в задаче протекания не выходит на стационар: в видео-моделях фиксируются, после некоторого времени протекания, внезапные всплески значений плотности и температуры. Чем объяснить такие всплески - недостатками схемы или же реальными физическими процессами - пока неясно. Однако поставленную задачу можно считать выполненной. 
\newpage
\addcontentsline{toc}{section}{Список литературы}%
\begin{thebibliography}{2}

\bibitem{Popov} Попов А.В., Численное моделирование нестационарного течения газа с использованием неявных разностных схем, 2022;

\bibitem{group} Бахвалов Н.С., Жидков Н.П., Кобельков Г.М., Численные методы, 2011; 
\end{thebibliography}

\end{document}

